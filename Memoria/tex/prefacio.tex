
\cleardoublepage
\thispagestyle{empty}

\begin{center}
{\large\bfseries Sistema de recuperación de imágenes basado en
términos lingüísticos de alto nivel semántico}\\
\end{center}
\begin{center}
Luis Suárez Lloréns\\
\end{center}

%\vspace{0.7cm}
\noindent{\textbf{Palabras clave}: imagen, recuperación de información, forma, lógica difusa, aprendizaje automático, deep learning, red neuronal, red neuronal convolucional.}\\

\vspace{0.7cm}
\noindent{\textbf{Resumen}}\\

En la actualidad, la cantidad de información generada por el ser humano está creciendo a un ritmo muy rápido. Por ejemplo, con la fotografía digital, que elimina las limitaciones del carrete, la cantidad de imágenes en las manos de los usuarios ya empezó a crecer a gran velocidad. Este fenómeno ha crecido aún más con el crecimiento de las redes sociales, donde los usuarios depositan sus comentarios, fotos y videos.

Este crecimiento genera un nuevo problema. El usuario tiene que ser capaz de encontrar su información entre tantos ficheros. La solución es un sistema de recuperación de información que sea capaz de encontrar el fichero del usuario tras una consulta del mismo. Los buscadores de web son un claro ejemplo de este tipo de sistemas.

Este proyecto intenta crear un sistema de recuperación de imágenes, pero con una pequeña modificación. Normalmente, los sistemas de recuperación de imágenes necesitan información aportada por el usuario para realizar la búsqueda. Un ejemplo es la plataforma Instagram, donde los propios usuarios clasifican sus fotos, y luego se pueden buscar imágenes gracias a esa información. El objetivo del proyecto será que la aplicación que sea capaz de clasificar automáticamente las imágenes, sin necesidad de interacción del usuario, para después utilizar esa información generada para ser capaz de recuperar las imágenes.

Normalmente, las aplicaciones existentes de recuperación de imágenes, que usan solo información de las imágenes para buscar, tienen definido un conjunto de descriptores de la imagen. Estos descriptores se comparan a los descriptores de la imagen de consulta, para recuperar imágenes que se parezcan a la de consulta.

El objetivo es entonces transformar imágenes en palabras, y va a tratar dos enfoques del problema. Por un lado, tratar de describir las formas de un objeto y por otro, utilizar técnicas de aprendizaje automático para clasificar automáticamente imágenes.


Para el primer objetivo, usaremos la lógica difusa. En el mundo real esta lleno de imprecisiones y de diferentes puntos de vista. por ejemplo, es común considerar la esquina de una mesa como un vértice, aunque seguramente esta estará redondeada. La lógica difusa es uno de los modelos matemáticos que intenta adaptar la lógica formal al mundo real, tratando de adaptarse y modelar las imprecisiones que se encuentran en el mundo real. Por tanto, usaremos la lógica difusa para definir conceptos básicos de los contornos: ``linealidad'', ``curvacidad'' y ``verticidad''. Estos elementos se podrán usar en el futuro para definir y poder reconocer formas más complejas, como por ejemplo polígonos.  

Para el segundo objetivo, utilizaremos las técnicas de aprendizaje automático que mejores resultados han dado en las competiciones de clasificación de imágenes. Utilizaremos un modelo encajado dentro del conocido deep learning. Tradicionalmente, en los problemas de aprendizaje automático, se generaba un conjunto de características, que luego se daban como entrada a un clasificador. Los modelos de deep learinig se caracterizan por utilizar como entrada al clasificador el dato sin procesar. Es por tanto, el propio clasificador el que aprende por su cuenta las características. El tipo que clasificador que utilizaremos será una red neuronal convolucional, un tipo de red neuronal artificial. Estos modelos imitan el comportamiento de un cerebro, donde una gran cantidad de neuronas artificiales se comunican entre ellas para tratar de obtener el resultado  deseado.

La aplicación utilizará las técnicas desarrolladas en los dos apartados para recuperar imágenes dentro del ordenador del usuario, de una manera sencilla, mostrándole al usuario los resultados ordenados según el nivel de satisfacción de la consulta.
 
\cleardoublepage


\thispagestyle{empty}

\selectlanguage{USenglish}
\begin{center}
{\large\bfseries Project Title: Project Subtitle}\\
\end{center}
\begin{center}
Luis Suárez Lloréns\\
\end{center}

%\vspace{0.7cm}
\noindent{\textbf{Keywords}: Image, Information retrieval, shape, fuzzy logic, machine learning, deep learning, neural network, convolutional neural network}\\

\vspace{0.7cm}
\noindent{\textbf{Abstract}}\\

Nowadays, the amount of information and files generated by the human kind is growing way faster than we can handle. A good example of this phenomenon is the digital photography. It changed the way people used their cameras, from a few pics in every travel to hundreds of photos per day given the memory increase and ease of use this technology. Now, it is really common to store thousand of pics in our computers.  There are more examples of this new tendency.  New social media networks, like Facebook, Twitter and Instagram are clear examples of this growth. Millions of users upload their opinions, photographies and videos to this social networks. 

A problem arise from this grow of information. The user is unable to find a file or picture, even in the user’s device. The solution is an Information Retrieval system  that finds the information after a query from the user. Web search engines are the most noticeable example of this kind of applications.

This project tries to create an Information Retrieval System for pictures and photos from the user’s device, with a little twist. Usually, Information Retrieval Systems need some information to be able to search. Web search engines use text, which is a really easy type of information to use for these purposes, because is a structured type of data. For pictures, some social media network like Instagram let their users tag the images, and after that uses those tags to search the database after a query . The objective of this project is to create an application that is able to automatically generate tags for the images of the user. Then, it will use that information to show images that fulfil the user’s query. Therefore, the application needs an Artificial Intelligence to be able to do the task. 

Usually, the Information Retrieval Systems for images that only uses the information from the images to search, have defined some set of descriptors like mean color, dominant colors, texture or edges of the image. Then, all those descriptors are compared, in order to show the closest images to the query image. So this kind of Information Retrieval System transform images into a vector of values for those descriptors.

This project objective is to transform images into words. It will focus on two aspects. The first one is to define qualities of the shape of the object in order to be able to describe the shape of the object. The second one is the use of machine learning techniques, like neural networks, to classify an image into a set of categories.

The shape of an object is a really good source of information to be able to recognize the object. With only a black silhouette, we can easily distinguish two different elements. For example, the shape of a heart or the shape of a fish is really easy to recognize. Therefore, the shape must be a good descriptor of our images. But images, and the perception of the images by the people is really imprecise. For example, a person can classify a table’s corner as a vertex, another person can classify it as a no-vertex, because is too rounded to be a vertex.

So, in order to be able to modelize this reality, is necessary to introduce a new kind of logic that accept the imprecision inherent of this problem. There are lots of mathematical models trying to solve this problem. In this project, we are going to use one of them, the fuzzy set theory and fuzzy logic. The fuzzy set theory is an extension to the usual set theory. There is a major difference between the two theories. In the usual set theory, an element is either on the set or outside the set. In the fuzzy set theory, an element have a degree of membership to the set. As an example, imagine a person which height is 1.80 meters. In the usual set theory, there are only two options. Either is a tall person or he is not. It leaves no place for “is no tall enough but is not short” kind of description. Inside the fuzzy set theory, is possible to make this kind of descriptions. For example, that person will have an 0.7 membership value in the set “tall” and a 0.3 in the set “short”. This ability creates a whole new world of possibilities.

This project will describe the shape of the objects through the use of fuzzy logic. It will define the concepts “linearity”, “curvacity” and “verticity”. Those concept will be use in the future to be able to describe shapes, starting with the most simples ones, like squares and any other polygons.

The other side of the project is the use of a classifier to automatically tag the images of the user. The objective is to use machine learning techniques, among a really big set of images, to be able to learn the paterns necesary to recognize the objects in them. The state of art classifier for this task is based directly in a brain. An artificial neural network is a simplified version of a brain. In then, hundreds or thousends of artificial neurons interconect with the other, in order to process the input information into a result.

Usually, this kinds of artificial intelligence applications needed the creation of a very well designed descriptors set. But in this case, the data will be feed raw to the artificial neural network, without the creation of any descriptor. Then the multiple layers of the neural network will learn really complex descriptors automatically.  Feeding the data raw to the artificial intelligence, leaving to the system the task of the creation of the descriptors is called deep learning.

An specific type of neural network, convolutional neural network, will be used for the application. This type of neural network, introduced by Yann Lecun, has made greats advances in the field of image classification and many other field, where the data is not structurated. The idea is to search features in a window, and pass that window through all the image. For example, to find an eye in a picture, we will use a window of the proper size to find that feature, instead of searching in the image as a whole. Using that idea, Lecun was able to reduce the input to every neuron, from all the pixels in the image to only the pixels inside the window. Even more, all the neurons in the same layer will share the feature they search, so they also share the weights they use to compute. This is logical because the system does not know where the object will be in the image, and if a feature is meaningful, the system will want to find it anywhere. These little adjustments, to fit the neural networks for their use with images was a big revolution, shattering the competitors in multiple competitions of machine learning.
\selectlanguage{spanish}
\chapter*{}
\thispagestyle{empty}

\noindent\rule[-1ex]{\textwidth}{2pt}\\[4.5ex]

Yo, \textbf{L}, alumno de la titulación TITULACIÓN de la \textbf{Escuela Técnica Superior
de Ingenierías Informática y de Telecomunicación de la Universidad de Granada}, con DNI XXXXXXXXX, autorizo la
ubicación de la siguiente copia de mi Trabajo Fin de Grado en la biblioteca del centro para que pueda ser
consultada por las personas que lo deseen.

\vspace{6cm}

\noindent Fdo: Luis Suárez Lloréns

\vspace{2cm}

\begin{flushright}
Granada a X de septiembre de 2016.
\end{flushright}


\chapter*{}
\thispagestyle{empty}

\noindent\rule[-1ex]{\textwidth}{2pt}\\[4.5ex]

D. \textbf{Jesús Chamorro Martínez}, Profesor del Área de XXXX del Departamento YYYY de la Universidad de Granada.

\vspace{0.5cm}

\textbf{Informa:}

\vspace{0.5cm}

Que el presente trabajo, titulado \textit{\textbf{Sistema de recuperación de imágenes basado en
términos lingüísticos de alto nivel semántico
\\}},
ha sido realizado bajo su supervisión por \textbf{Luis Suárez Lloréns}, y autoriza la defensa de dicho trabajo ante el tribunal
que corresponda.

\vspace{0.5cm}

Y para que conste, expide y firma el presente informe en Granada a X de septiembre de 2016.

\vspace{1cm}

\textbf{El director:}

\vspace{5cm}

\noindent \textbf{Jesús Chamorro Martínez}

\chapter*{Agradecimientos}
\thispagestyle{empty}

       \vspace{1cm}


Poner aquí agradecimientos...

