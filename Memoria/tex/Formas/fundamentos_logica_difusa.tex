\section{Fundamentos de la lógica difusa}
\todo[inline,color=red!50]{Citar a Belén}
Es común pensar que una persona que mide un metro con ochenta y cinco centímetros es una persona alta. Pero, ¿cuándo una persona deja de ser alta? ¿Ese cambio es brusco? ¿Todas las personas tienen la misma clasificación de persona alta? En el razonamiento humano usual, no todos los hechos son ciertos o falsos, si no que hay matices e incluso diferentes puntos de vista. Se puede considerar que la información puede ser:

\begin{itemize}
\item \textbf{Incompleta:} Describe parcialmente la realidad.
\item \textbf{Imprecisa:} El valor de una variable se encuentra en un conjunto de valores pero no podemos precisar cual es.
\item \textbf{Incierta:} No tenemos total certeza de que la información sea verdadera.
\end{itemize}

Aún así, el ser humano utiliza estos conceptos imprecisos para razonar, proporcionando a las personas una gran capacidad de razonamiento, y de adaptar ese razonamiento a la situación.\\

Han habido varios intentos de modelizar este fenómeno. Entre ellos, encontramos la Teoría de Conjuntos Difusos, formulada por Lofti A. Zadeh en 1965\cite{Zadeh-Fuzzy-1965}. Este matemático e ingeniero azerbaiyano, basó la lógica difusa en la teoría de conjuntos ordinaria, añadiendo a esta un grado de pertenencia de los elementos al conjunto, en vez de la información binaria de los conjuntos ordinarios, es decir, si un elemento pertenece o no al conjunto sin niveles intermedios.\\

Esta breve introducción a la lógica difusa de Zadeh describirá los conceptos necesarios para poder definir en la siguiente sección propiedades difusas sobre las formas. En concreto se tratará la definición de conjunto difuso y las operaciones básicas con conjuntos difusos.\\

\subsection{Conjuntos difusos}

Un conjunto difuso sobre un universo X es una generalización del concepto clásico de conjunto en el que la función que indica si un elemento pertenece o no al conjunto, la función indicadora, tiene como rango el intervalo real [0,1], en vez del conjunto {0,1}. Por tanto, el conjunto difuso $A$ viene descrito por la función indicadora $\mu_A$:

\[
\ \mu_A : X \longrightarrow [0,1].
\]

En el contexto de la teoría de subconjuntos difusos, se denomina función de pertenencia a la función indicadora. Por lo general, se suele identificar al conjunto difuso con su función de pertenencia. Seguiremos la siguiente notación:

\[
\ A : X \longrightarrow [0,1]
\]

donde, siendo $x \in X$, $A(x)$ representa el grado de pertenencia del elemento $x$ al conjunto difuso $A$.\\

Algunos conceptos básicos sobre conjuntos difusos son:

\begin{itemize}
\item Un conjunto difuso $A$ se dice \textit{normal} si existe al menos un $x \in X$ tal que $A(x)=1$.
\item Se llama \textit{soporte} del conjunto difuso $A$ al conjunto

\[
\ Sop(A) = \lbrace x \in X | A(x) > 0 \rbrace.
\] 
\item Se llama \textit{núcleo} del conjunto difuso $A$ al conjunto

\[
\ Ker(A) = \lbrace x \in X | A(x) = 1 \rbrace.
\] 
\end{itemize}

\subsection{Operaciones básicas con conjuntos difusos}

La extensión de los conjuntos usuales a conjuntos difusos no tendría sentido sin la existencia de operadores básicos para los conjuntos difusos. Por tanto, se pretende extender también de las operaciones principales sobre conjuntos. Las extensiones pueden realizarse de diversas formas, con la condición de que deben comportarse como los operadores ordinarios cuando los conjuntos implicados son ordinarios. Las familias de operadores difusos más importantes son las \textit{t}-normas que extienden a las uniones, las \textit{t}-conormas, que extienden a la unión y las negaciones, que extienden al complemento.\\

\subsubsection{Operadores de intersección: \textit{t}-normas}

Una \textit{t}-norma es una función

\[
\ i: [0,1] \times [0,1] \longrightarrow [0,1]
\]

que verifica las siguientes propiedades para todo $a,b,c \in [0,1]$:

\begin{itemize}
\item \textbf{Frontera:} $i(a,1) = a$.
\item \textbf{Monotonía:} $b \leq c \Rightarrow i(a,b) \leq i(a,c)$.
\item \textbf{Conmutatividad:} $i(a,b) = i(b,a)$.
\item \textbf{Asociatividad:} $i(a,i(b,c)) = i(i(a,b),c)$.
\end{itemize}

Dentro de esta familia de funciones, las más utilizadas usualmente son:

\begin{itemize}
\item \textbf{Mínimo:} $i(a,b)= min(a,b)$.
\item \textbf{Producto algebraico:} $i(a,b)= ab$.
\item \textbf{Recta acortada (Lukasiewicz):} $i(a,b)= max(0,a+b-1)$.
\item \textbf{Intersección drástica:} $i(a,b)= \left\lbrace
  \begin{array}{l}
     b \qquad a = 1 \\
     a \qquad b = 1 \\
     0 \qquad \textrm{ en otro caso }
  \end{array}
  \right.$.
\end{itemize}

Gracias a cualquiera de las \textit{t}-normas, podemos definir la intersección de dos conjuntos difusos $A$ y $B$ de la siguiente forma:

\[
\ (A\cap B)(x)= i(A(x),B(x)).
\]

\subsubsection{Operadores de unión: \textit{t}-conormas}

Una \textit{t}-conorma es una función

\[
\ u: [0,1] \times [0,1] \longrightarrow [0,1]
\]

que verifica las siguientes propiedades para todo $a,b,c \in [0,1]$:

\begin{itemize}
\item \textbf{Frontera:} $u(a,0) = a$.
\item \textbf{Monotonía:} $b \leq c \Rightarrow u(a,b) \leq u(a,c)$.
\item \textbf{Conmutatividad:} $u(a,b) = u(b,a)$.
\item \textbf{Asociatividad:} $u(a,u(b,c)) = u(u(a,b),c)$.
\end{itemize}

Dentro de la familia de funciones de las \textit{t}-conormas, las más utilizadas usualmente son:

\begin{itemize}
\item \textbf{Máximo:} $i(a,b)= max(a,b)$.
\item \textbf{Suma algebraica:} $u(a,b)= a+b-ab$.\todo[color=blue!40]{Sure?}
\item \textbf{Suma acortada (Lukasiewicz):} $u(a,b)= min(1,a+b)$.
\item \textbf{Intersección drástica:} $u(a,b)= \left\lbrace
  \begin{array}{l}
     b \qquad a = 0 \\
     a \qquad b = 0 \\
     1 \qquad \textrm{ en otro caso }
  \end{array}
  \right.$.
\end{itemize}

Gracias a cualquiera de las \textit{t}-conormas, podemos definir la unión de dos conjuntos difusos $A$ y $B$ de la siguiente forma:

\[
\ (A \cup B)(x)= u(A(x),B(x)).
\]

\subsubsection{Operadores de complemento: negaciones}

Una negación es una función

\[
\ c: [0,1] \longrightarrow [0,1]
\]

que verifica las siguientes propiedades para todo $a,b \in [0,1]$:
\begin{itemize}
\item \textbf{Frontera:} $c(0) = 1$ y $c(1) = 0$.
\item \textbf{Monotonía:} $a \leq b \Rightarrow c(b) \leq c(a)$.
\end{itemize}

Dentro de la familia de funciones de las \textit{t}-conormas, las más utilizadas usualmente son:

\begin{itemize}
\item \textbf{Negación estándar:} $c(a)= 1-a$.
\item \textbf{Negación umbral:} $c(a)= \left\lbrace
  \begin{array}{l}
     1 \qquad a \geq t \\
     0 \qquad a < t \\
  \end{array}
  \right. , \textrm{con } t\in (0,1]$.
\end{itemize}

Gracias a cualquiera de las negaciones, podemos definir el complemento de un conjunto difusos $A$ de la siguiente forma:

\[
\ \bar{A}(x)= c(A(x)).
\]

Normalmente, dentro de las negaciones, se elige la negación estándar por sus buenas propiedades, como por ejemplo ser continua e involutiva, es decir, $c\left(c\left(a\right)\right)=a$, para todo $a \in [0,1]$ .

\subsubsection{Propiedades de los operadores estándar}

El operador estándar de intersección, el mínimo, tiene la interesante propiedad de producir la mayor la función de pertenencia de todas las \textit{t}-normas \cite{Ross}. De modo similar, el operadodr estándar de unión, el máximo, tiene la propiedad de producir la menor función de pertenencia de todas las \textit{t}-conormas. Estas propiedades tienen su importancia ya que previenen del empeoramiento de los errores en los operandos. La mayoría de otras \textit{t}-normas y \textit{t}-conormas no tienen esta ventaja.\\

Utilizando los operadores estándar podemos encontrar propiedades adicionales, similares a las que cumplen los conjuntos ordinarios. Destacar la propiedad distriburiva y las leyes de De Morgan. Siendo $A$, $B$ y $C$ conjuntos difusos:

\begin{itemize}
\item \textbf{Leyes de De Morgan:}
\[
\ \overline{A \cup B} = \overline{A} \cap \overline{B}
\]
\[
\ \overline{A \cap B} = \overline{A} \cup \overline{B}
\]
\item \textbf{Propiedad distributiva:}
\[
\ C \cap \left( A \cup B \right) = \left(C \cap A \right) \cup \left(C \cap B \right)
\]
\[
\ C \cup \left( A \cap B \right) = \left(C \cup A \right) \cap \left(C \cup B \right)
\]
\end{itemize}

Pero no podemos obtener todas las propiedades usuales de los conjuntos ordinarios. Por ejemplo, siendo $A$ un conjunto difuso, las propiedades

\[
\ A \cup \overline{A} = X
\]
\[
\ A \cap \overline{A} = \emptyset
\]

no se cumplen. Dado $x \in X$ y supongamos que $A(x)=\frac{1}{2}$, se cumple $\left( A \cup \overline{A} \right) (x) = \left( A \cap \overline{A} \right) (x) = A(x)$. Por tanto, tenemos un contrajemplo, que nos hace descartar las propiedades anteriores.\\

\subsection{Modificadores lingüísticos}

En el habla, utilizamos adjetivos que alteran la intensidad de una categoría. Por ejemplo, ``muy altos'' modifica la categoría ``altos'' en pos de hacerla más restrictiva, quedándonos únicamente con las personas realmente altas. En la lógica difusa, se crean unos operadores, llamados modificadores lingüísticos, que permiten adaptar esta realidad, realzando valores de la función de pertenencia o haciéndola más o menos permisiva.\\

Se pueden definir muchos modificadores lingüísticos. Vamos a mostrar algunos de los más básicos. Sea $\mu_A$ la función de pertenencia de un conjunto difuso, se definen los siguientes modificadores lingüísticos:

\begin{itemize}
\item \textbf{``Muy'': }$\mu_A = \mu_A^2$.
\item \textbf{``Muy muy'': }$\mu_A = \mu_A^4$.
\item \textbf{``Más'': }$\mu_A = \mu_A^{1.25}$.
\item \textbf{`Un poco'': }$\mu_A = \sqrt{\mu_A}$.
\item \textbf{``Menos'': }$\mu_A = \mu_A^{0.75}$.
\end{itemize}

Los tres primeros forman parte de los modificadores lingüísticos de concentración. Se llaman así ya que reducen mucho la función de pertenencia de los elementos que tienen una baja pertenencia al conjunto, dejando casi sin alterar los elementos que tienen un alto valor de función de pertenencia, consiguiendo un efecto de concentración en torno a los valores que mejor cumplen la condición del conjunto difuso. De manera opuesta, los dos últimos modificadores lingüísticos se conocen como de dilatación.\\

En la teoría de conjuntos difusos podemos encontrar muchos más modificadores lingüísticos y categorías de los mismos.\\
