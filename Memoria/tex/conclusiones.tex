\section{Conclusiones}

La construcción de este tipo de aplicaciones, tiene un gran futuro y seguro que en poco tiempo, grandes empresas del sector lanzarán funcionalidades de este estilo. Quizás, el mayor problema lo encontramos en las técnicas de aprendizaje automático que, pese a ser muy potentes, todavía son débiles y no llegan al nivel de un humano en la clasificación de imágenes. Esta falta de un porcentaje de acierto muy muy elevado hace que aún sea un poco pronto y aventurado su uso en una aplicación comercial. Un fallo en este tipo de aplicaciones puede llegar a ser considerado incluso ofensivo por algunos usuarios. Esta limitación puede ser la causa del freno de estas aplicaciones en un entorno general.

El proyecto ha conseguido finalizar los objetivos planteados en un primer momento. Se ha llevado a cabo un estudio de los fundamentos teóricos de los dos modelos utilizados, y estos se han utilizado como métodos de obtener información automáticamente de una imagen. La aplicación desarrollada es un pequeño prototipo, capaz de recuperar información utilizando para ellos un amplio abanico de términos lingüísticos de alto nivel. La extracción de información para la clasificación se realiza mediante una red neuronal convolucional, una de los modelos que mejor resultado han conseguido en el campo de la clasificación automática de imágenes.\\

El objetivo añadido de crear un sistema basado en la lógica difusa para clasificar formas en base a términos lingüísticos ha sido alcanzado parcialmente. Si bien no se ha llegado a completar su desarrollo, se han planteado las bases para su finalización en el futuro y se ha usado como un descriptor para el prototipo.\\


\section{Trabajo Futuro}

La aplicación desarrollada es de un carácter simple, y podría ser mejorada en varios aspectos: Por una parte, se podría ver beneficiada de una mejora a nivel visual, mejorando la interfaz para el usuario. Por otro lado, se podría aumentar la funcionalidad, aumentando por ejemplo el número de clasificadores integrados en la misma o el número de características por las que buscar.\\
\begin{itemize}
\item \textbf{Gestión de bases de Datos:} El método de almacenamiento de la información implementado, la serialización directa de la información, es demasiado simple. Para conseguir que el sistema sea escalable y mejorar las búsquedas y el almacenamiento, se debería de guardar dicha información dentro de una base de datos adecuada.
\item \textbf{Dependencia de Caffe:} La fuerte dependencia del prototipo a la librería Caffe hace que este sistema sea difícil de instalar. Eliminar esta dependencia por otra que tenga una mejor integración en el lenguaje Java o eliminar totalmente la dependencia con la creación de un módulo propio sería una gran mejora al facilitar el uso del sistema para un nuevo usuario.
\item \textbf{Uso de Tesauros u Ontologías:} Las búsquedas asociadas a conceptos tienen una flexibilidad limitada. La inclusión de este tipo de sistemas podría conseguir un mejor abanico de conceptos con los cuales realizar las consultas, consiguiendo un sistema de recuperación más flexible.
\end{itemize}

Por supuesto, la creación y diseño de nuevos clasificadores que mejoren los resultados del clasificador de la aplicación es también una vía de futuro. Este es un campo donde varias empresas fuertes del sector están realizando grandes esfuerzos. La investigación en este campo sigue siendo importante y se encuentra actualmente al alza. \\

En cuanto a los descriptores de formas, queda mucho trabajo por hacer. Hay que hacer un estudio más profundo de distintas variables, por ejemplo el tamaño de ventana y buscar técnicas para adaptarlo automáticamente según la figura. Además hay que avanzar en la descripción de formas a partir de ellos, para poder llegar al objetivo de transformar una forma en conceptos del tipo triángulo, cuadrado, etcétera. Incluso, se podría hacer una comparación con las medidas usuales, por ejemplo la curvatura, para ver si se mejora el funcionamiento de técnicas tales como la búsqueda de puntos de interés y la reconstrucción de las imágenes a partir de dichos puntos y la información que tenemos de ellos.\\
