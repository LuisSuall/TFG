\section{Conclusiones}

\section{Trabajo Futuro}

La aplicación desarrollada es de un carácter simple, y podría ser mejorada en varios aspectos. Por una parte, se podría ver beneficiada de una mejora a nivel visual, mejorando la interfaz para el usuario. Por otro lado, se podría aumentar la funcionalidad, aumentando por ejemplo el número de clasificadores integrados en la misma o el número de características por las que buscar.\\

Además, el método de almacenamiento de la información es bastante simple. Para conseguir que el sistema sea escalable, se debería de guardar dicha información dentro de una base de datos adecuada, para facilitar las búsquedas y el almacenamiento.\\

Por supuesto, la definición de clasificadores que mejoren los resultados del clasificador de la aplicación es también una vía de futuro. Este es un campo donde varias empresas fuertes del sector están realizando grandes esfuerzos. La investigación en este campo sigue siendo importante y se encuentra actualmente al alza.\\

En cuanto a los descriptores de formas, queda mucho trabajo por hacer. Hay que hacer un estudio más profundo de distintas variables, por ejemplo el tamaño de ventana y buscar técnicas para adaptarlo automáticamente según la figura. Además hay que avanzar en la descripción de formas a partir de ellos, para poder llegar al objetivo de transformar una forma en conceptos del tipo triángulo, cuadrado, etcétera. Incluso, se podría hacer una comparación con las medidas usuales, por ejemplo la curvatura, para ver si se mejora el funcionamiento de técnicas tales como la búsqueda de puntos de interés y la reconstrucción de las imágenes a partir de dichos puntos y la información que tenemos de ellos.\\

