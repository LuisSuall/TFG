La gestión de la información generada por los seres humanos es una de las tareas fundamentales de la informática. Además de las tareas básicas esperadas de este proceso, como el almacenamiento y la edición de dicha información, se plantea también la necesidad de clasificar esta información, para facilitar el acceso rápido y eficiente a la misma.\\

Por supuesto, la clasificación puede ser realizada manualmente por los usuarios, pero este proceso se puede volver intratable cuando la cantidad de información crece de manera muy rápida. Por tanto, es necesario automatizar el proceso, y dejar que el ordenador ordene por categorías la información.\\

Esta tarea, que es compleja por sí misma, se dificulta más cuando la información no está estructurada de manera clara. El caso de estudio de este trabajo se centra en el mundo de las imágenes. Es fácil ver que, salvo algunos casos donde las imágenes se toman en un ambiente altamente controlado ---imágenes de control de calidad en una fábrica, por ejemplo---, las imágenes no tienen una estructura, ni áreas a destacar de antemano. Es más, si en la imagen aparece un animal para una persona es muy fácil reconocerlo, pero es imposible que sea capaz de definir con precisión que hace que esos píxeles representen un animal.\\

Aún así, evidentemente las imágenes son una de las fuentes fundamentales de información utilizada por el ser humano, y además con el crecimiento de Internet y las redes sociales, así como la popularización de los teléfonos inteligentes, el número de imágenes a crecido a un ritmo impresionante. Por tanto, la clasificación automática de las imágenes es una tarea imprescindible.\\

Por supuesto, este problema ha sido tratado con anterioridad y sigue siendo un campo de investigación en continuo desarrollo. Podríamos dividir los avances en la clasificación automática de las imágenes en tres etapas bien diferenciadas.\\

Por un lado, tenemos la primera etapa, en ella... (etiquetado automático usando texto próximo)\todo[color=red!50]{Terminar introducción}

En la segunda etapa, el enfoque cambió... (uso de descriptores automáticos).

En la tercera... (clasificación automática)

Este trabajo consta de dos partes bien diferenciadas.\\

La primera estará centrada en el estudio de la forma de los objetos, como un posible descriptor usado en la segunda generación de la clasificación automática de imágenes. Claramente, la forma es un factor de gran importancia para reconocer los objetos. Se estudiará un descriptor clásico como es el de la curvatura del contorno, y se mostrará los avances realizados durante la investigación realizada como parte de la Beca de Iniciación a la Investigación de la Universidad de Granada para tratar de analizar las formas con el uso de descriptores lingüísticos, utilizando para ello la lógica difusa.\\

La segunda parte estará centrada en el estudio del estado del arte de la tercera generación de clasificación automática de imágenes. Este campo utiliza técnicas avanzadas del área Machine Learning o Aprendizaje Automático. Por tanto, se realizará una breve introducción al campo del Machine Learning, Deep Learning y los diversos modelos que llevan hasta el utilizado actualmente para la clasificación automática de imágenes, las redes neuronales convolucionales.\\

\section{Objetivos}
