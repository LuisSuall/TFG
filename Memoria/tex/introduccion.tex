La gestión de la información generada por los seres humanos es una de las tareas fundamentales de la informática. Además de las tareas básicas esperadas de este proceso, como el almacenamiento y la edición de dicha información, se plantea también la necesidad de clasificar esta información, para facilitar el acceso rápido y eficiente a la misma.\\

Por supuesto, la clasificación puede ser realizada manualmente por los usuarios, pero este proceso se puede volver intratable cuando la cantidad de información crece de manera muy rápida. Por tanto, es necesario automatizar el proceso, y dejar que el ordenador ordene por categorías la información.\\

Esta tarea, que es compleja por sí misma, se dificulta más cuando la información no está estructurada de manera clara. El caso de estudio de este trabajo se centra en el mundo de las imágenes. Es fácil ver que, salvo algunos casos donde las imágenes se toman en un ambiente altamente controlado, como por ejemplo las imágenes de control de calidad en una fábrica, las imágenes no tienen una estructura, ni áreas a destacar de antemano. Es más, si en la imagen aparece un animal para una persona es muy fácil reconocerlo, pero es imposible que sea capaz de definir con precisión que hace que esos píxeles representen un animal.\\

Aún así, evidentemente las imágenes son una de las fuentes fundamentales de información utilizada por el ser humano, y además con el crecimiento de Internet y las redes sociales, así como la popularización de los teléfonos inteligentes, el número de imágenes a crecido a un ritmo impresionante. Por tanto, la clasificación automática de las imágenes es una tarea imprescindible.\\

Por supuesto, este problema ha sido tratado con anterioridad y sigue siendo un campo de investigación en continuo desarrollo. Podríamos dividir los avances en la clasificación automática de las imágenes en tres etapas bien diferenciadas.\\

Por un lado, en los primeros modelos se hacía recuperación basada en texto. Normalmente, en especial en internet, las imágenes se encuentran inmersas en un texto escrito por los propios usuarios. Por tanto, se utilizaban las técnicas de recuperación basadas en texto, mucho más desarrolladas, para recuperar imágenes. Estos modelos aceptan palabras como consulta, pero no considera ninguna información de la imagen.\\

En un segundo enfoque, se trató de utilizar la información de las imágenes, por medio de descriptores estadísticos simples, como por ejemplo el color medio. Estos descriptores se usaban tras su extracción para la comparación con otras imágenes. El pequeño defecto de estos modelos es que la entrada tienen que ser los datos de dichos descriptores o en su defecto imágenes de ejemplo para que sea el sistema el que calcule los descriptores.\\

Una nueva vertiente trata de obtener descripciones en base a conceptos de las imágenes. Es decir que tras analizar la imagen, nos indicaría que, por ejemplo, un objeto es cuadrado o a un más alto nivel, que el objeto es una pizarra. Estos modelos utilizan sólo la información de la propia imagen, pero la transforma en descriptores lingüísticos, más fáciles de comprender y de usar para el usuario.\\

El proyecto se basará en la construcción de un sistema CBIR (Content Based Image Retrieval) que use términos lingüísticos para la descripción y consulta de imágenes. Se tratarán de usar dos aproximaciones para la generación de dichos descripciones, utilizando siempre como única fuente de información la propia imagen.\\

La primera estará centrada en el estudio de la forma de los objetos. Claramente, la forma es un factor de gran importancia para reconocer los objetos y permite al ser humano reconocer elementos sin necesitar ninguna información adicional. Tras una breve introducción a la lógica difusa\cite{Zadeh-Fuzzy-1965}\cite{Ross}, se mostrará los avances realizados durante la investigación realizada como parte de la Beca de Iniciación a la Investigación de la Universidad de Granada para tratar de analizar las formas con el uso de descriptores lingüísticos, utilizando para ello la lógica difusa.\\

La segunda estará centrada en el estado del arte de la clasificación automática mediante el uso de técnicas de deep learning que han mostrado buenos resultados en la clasificación de imágenes. Este campo utiliza técnicas avanzadas del área Machine Learning o Aprendizaje Automático. Por tanto, se realizará una breve introducción a la problemática del aprendizaje automático \cite{Abu-Mostafa:2012:LD:2207825}, y se desarrollarán más en profundidad los diversos modelos que llevan hasta el utilizado actualmente para la clasificación automática de imágenes, las redes neuronales convolucionales\cite{Bishop:2006:PRM:1162264}\cite{lecun-89e}.\\


\section{Objetivos}

En la propuesta inicial se propusieron los siguientes objetivos:

\begin{enumerate}
\item Estudiar las técnicas de aprendizaje automático (deep learning), y los modelos matemáticos subyacentes, adecuadas para el etiquetado de imágenes.
\item Analizar y seleccionar bases de datos de imágenes previamente etiquetadas con términos lingüísticos de medio/alto nivel.
\item Aplicar la(s) técnica(s) seleccionada(s) al aprendizaje automático de un conjunto de términos.
\item Desarrollar un prototipo CBIR basado en los descriptores anteriores.
\end{enumerate}

A estos objetivos, y en el marco de la Beca de Iniciación a la Investigación, se añadió un objetivo más:
\begin{enumerate}
  \setcounter{enumi}{4}
  \item Creación de un sistema basado en la lógica difusa para incorporar conceptos relativos a la forma al prototipo CBIR.\\
\end{enumerate}

El desarrollo informático, que cubre los puntos 2, 3 y 4, se encuentra descrito en el capítulo \ref{ch3}, mientras que los apartados más teóricos se encuentran divididos en dos capítulos, el capítulo \ref{ch1} con el modelado matemático de las propiedades de las formas mediante el uso de lógica difusa y el capítulo \ref{ch2} con la definición del modelo matemático y el estudio de las técnicas propias del aprendizaje automático para el etiquetado de imágenes.\\