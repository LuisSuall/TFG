\section{Fundamentos del Aprendizaje Automático}

Pese a que es fácil reconocer un árbol en una imagen, la descripción precisa de que es un árbol es prácticamente imposible. La tarea se hace aún más difícil cuando tenemos que especificar estos detalles para una computadora. Aún así, se ha conseguido completar esta tarea, gracias al Aprendizaje Automático.\\

El Aprendizaje Automático es la rama de la informática que trata el estudio y la creación de algoritmos que puedan aprender automáticamente de los datos.\\

En esta sección seguiremos fundamentalmente el libro de Abu-Mostafa \cite{Abu-Mostafa:2012:LD:2207825} para la introducción a los conceptos más importantes del Aprendizaje Automático.\\

\subsection{Tipos de Aprendizaje Automático}

El Aprendizaje Automático se suelen dividir en tres tipos distintos:

\begin{itemize}
\item \textbf{Aprendizaje supervisado:}
\item \textbf{Aprendizaje por refuerzo:}
\item \textbf{Aprendizaje no supervisado:}
\end{itemize}

\subsection{Componentes del Aprendizaje Automático}
\subsection{¿Es posible aprender?}
\subsection{Error y ruido}
\subsection{Medidas de bondad del aprendizaje}
\subsection{Sobreaprendizaje}