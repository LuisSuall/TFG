Durante el desarrollo, se ha tenido en cuenta que la aplicación tiene que ser fácil de usar por el usuario. Si bien es un prototipo sencillo, siempre es necesaria una explicación sobre el uso básico del mismo. En este pequeño manual, vamos a ver cómo utilizar el prototipo desarrollado, así como mostrar ejemplos de uso del mismo. 

\section{Ventana principal}

\section{Ventana de imagen}

\section{Ventana de clasificación}

\section{Ventana de descriptor}

\section{Ventana de resultados de búsqueda}

\section{Acciones}

Vamos a describir las acciones que puede realizar el usuario:
\begin{itemize}
\item \textbf{Abrir imagen: }Para abrir una imagen, primero se clica sobre el botón ``abrir imagen''(1). Aparecerá un menú donde se deberá seleccionar la imagen deseada. Tras esto se obtendrá como resultado una ventana con la imagen seleccionada.
\item \textbf{Abrir base de datos:}Para abrir una base de datos, primero se clica sobre el botón de ``abrir base de datos'' (2). Aparecerá un menú donde se deberá seleccionar la base de datos deseada. Tras esto, se cargará la base de datos en la aplicación, indicándose así en la barra de estado.
\item \textbf{Calcular descriptor: } Teniendo una imagen activa. El usuario clica sobre el botón  ``Cálculo de descriptor'' y obtiene como resultado la ventana de dicho descriptor. El usuario puede hacer click derecho sobre el botón ``Cálculo de descriptor'' para cambiar el tipo de descriptor a calcular.
\item \textbf{Calcular clasificación: } Teniendo una imagen activa.  El usuario clica sobre el botón  ``Cálculo de clasificación'' y obtiene como resultado la ventana de clasificación asociada.
\item \textbf{Comparar descriptor con la base de datos: } Con una base de datos de tipo ``descriptor'' activa y una ventada de descriptor seleccionada, se clica en el botón de comparar. Tras esto, el usuario obtendrá una ventana con los resultados de la búsqueda.
\item \textbf{Buscar un concepto en la base de datos: } Con una base de datos de tipo ``clasificación'' activa, se introduce en el campo de texto la palabra a buscar y se clica en el botón de búsqueda o se presiona la tecla ENTER. Tras esto, se le mostrará al usuario una ventana con los resultados de su búsqueda.
\end{itemize}