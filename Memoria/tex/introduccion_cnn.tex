El etiquetado automático de las imágenes es un ámbito de la informática en pleno desarrollo. Pese a que se usan técnicas propias del campo de la visión por computador, los mayores avances se están obteniendo mediante la incorporación de técnicas de aprendizaje automático.\\

Estos avances se deben al desarrollo del llamado ``Deep Learning'', que ha permitido a los investigadores alcanzar buenos resultados en la clasificación en grandes bases de imágenes, como Image-net ~\cite{imagenet_cvpr09}, utilizando un modelo de redes neuronales convolucionales.\\

Se hace necesario por tanto un estudio de las bases del aprendizaje automático y de su problemática para después explicar el funcionamiento del ``Deep Learning'' y el modelo de las redes neuronales convolucionales. Tras esto, se mostrarán los resultados obtenidos con las redes neuronales convolucionales que estarán disponibles en la aplicación final en el desafío lanzado desde Image-net, ImageNet Large Scale Visual Recognition Competition (ILSVRC)~\cite{ILSVRC15}.\\

